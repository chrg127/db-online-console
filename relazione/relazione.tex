\documentclass[a4paper,12pt]{report}
\usepackage{alltt, fancyvrb, url}
\usepackage{graphicx}
\usepackage[utf8]{inputenc}
\usepackage{hyperref}
\usepackage[italian]{babel}
\usepackage[italian]{cleveref}
\usepackage[margin=0.5in]{geometry}
\usepackage{scrextend}
\title{Progetto di Database}
\author{Christian Ricci}
\date{\today}

\begin{document}
\maketitle
\tableofcontents

\chapter{Introduzione}



\chapter{Analisi dei requisiti}

Si vuole realizzare un database per gestire un servizio online di videogiochi. Il database dovrà immagazzinare dati relativi ai videogiochi, agli utenti, alle iscrizioni. I direttori del servizio potranno aggiungere nuovi videogiochi, consultare alcune statistiche, etc.

\section{Definizione delle specifiche in linguaggio naturale}

Il testo ottenuto dall'intervista con l'azienda è il seguente:

\begin{addmargin}[4em]{4em}
La casa produttrice di videogiochi GameSoft vuole realizzare un servizio online per tutti i videogiochi che ha pubblicato in passato. Gli utenti che si registreranno a questo nuovo servizio potranno accedere ad un catalogo di videogiochi e potranno giocare liberamente a qualunque gioco nel catalogo senza ulteriori costi. Degli utenti si vuole tener traccia del nome utente, della password, dell'email, dell'età e anche di un numero di telefono (opzionale).

Al momento della registrazione l'utente potrà scegliere tra tre piani: il primo prevede un periodo gratuito di un mese, che non si potrà rinnovare; a quel punto l'utente deve decidere tra gli altri due piani. Il secondo piano prevede un costo e comporta l'iscrizione al servizio per un mese. Il terzo piano, invece, comporta l'iscrizione per un anno. Si vuole anche memorizzare lo storico delle iscrizioni per ogni utente.

Per ogni videogioco si vuole memorizzare il titolo, il genere, l'anno di rilascio, l'azienda sviluppatrice e il produttore esecutivo. Per ogni utente del servizio si vuole anche dare la possibilità di acquistare una copia fisica di qualunque videogioco, che, però, potrebbe anche non essere disponibile.

Per ogni utente si vuole anche memorizzare alcune particolare statistiche: il numero di ore trascorso su un videogioco, il numero di ore trascorse in totale a giocare, una lista dei giochi più giocati, una lista dei giochi preferiti (a scelta dell'utente). Queste statistiche potranno essere viste dai direttori del servizio.

Infine, si vuole anche immagazzinare anche alcune informazioni relative al multiplayer. Gli utenti del servizio possono iniziare una sessione di gioco con altri utenti e scegliere un videogioco da giocare: si vuole memorizzare i dati di ogni sessione, che includeranno gli utenti partecipanti, il videogioco scelto e altre statistiche quali il tempo trascorso. Si vuole anche mantenere uno storico delle sessioni per ogni utente.

\end{addmargin}

\chapter{Progetto dello schema concettuale}

\chapter{Specifiche funzionali}

\chapter{Progetto logico}

\chapter{Progetto fisico}

\chapter{Interfaccia utente}

\chapter{Glossario}

\end{document}